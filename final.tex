\documentclass[letterpaper,12pt]{article}

\usepackage[table]{xcolor} % Must come before packages that load xcolor

\usepackage{amsfonts}
\usepackage{amsmath}
\usepackage{amssymb}
\usepackage{array}
\usepackage{booktabs}
\usepackage{color}
\usepackage{enumerate}
\usepackage{graphics}
\usepackage{mdwlist}
\usepackage{multicol}
\usepackage[all]{xy}

\usepackage[normalem]{ulem}
\usepackage[margin=1in]{geometry}
\setlength{\parskip}{1ex}

\newcommand{\la}{\langle}
\newcommand{\ra}{\rangle}
\newcommand{\lra}{\longrightarrow}
\newcommand{\tb}{\text{\textbullet}}
\newcommand{\sgn}{\text{sgn}}
\newcommand{\Tr}{\text{Tr}}
\newcommand{\im}{\text{im}}
\newcommand{\ol}{\overline}

\begin{document}

\title{Fixed Point Theorems and Evasiveness}
\author{Christopher Sumnicht}
\maketitle

\section{Introduction}

In the previous paper, we have concluded that for any simplex associated with a graph property to be nonevasive it must be collapsible. It may still be the case that there is an evasive property which is also collapsible. What then does this property buy us?

We can ask the question: Are there any graph properties which are not collapsible? This is still very much an active field of research, we will focus on one theorem in particular in this paper:

\begin{quote}
    Any nontrivial monotone graph property on a graph with a prime power of vertices is evasive.
\end{quote}

What is the general idea behind such a proof? Show that the associated simplicial complex is not collapsible!

\section{Simplicial Complexes and Algebra}

Before we begin to tackle decision trees we will have to make a detour and develop some tools for working with simplicial complexes. Here is a summary of what we will cover:

\begin{enumerate}
    \item{
            Definition of a Homology Group
        }
    \item{
            Definition of $F_p$ acyclic
        }
    \item{
            $\text{Collapsible} \implies F_p  \text{acyclic}$
        }
\end{enumerate}

\section{Collapsibility and Holes}

We first should provide some intuition as to why collapsibility can (although not nearly as well as homology groups) detect holes in our structure. Imagine that the simplicial complex is a compressible fluid and that the space that is not an element of the complex is solid and rigid. Imagine taking your hand and squishing the fluid in order to reduce it to a single point. If there is a hole, it will feel like a rock in your hand. Otherwise, you can continue compressing everything down to a single point.

Of course, this is simply an analogy. Let's actually prove it.

\subsection{Boundary Maps}

Suppose we want to get the boundary of a simplicial complex. For example if we have a filled triangle $\Delta$ we may want some operator $\partial$ that $\partial \Delta$ gives us just the edges. In the case, a filled triangle should become an empty triangle.

\begin{quote}
    We define the \textbf{Boundary Map} of a simplicial complex to be precisely:
    
    $$d_n([v_0,v_1,\ldots,v_n]) = \sum_{i=0}^n (-1)^i[v_0,v_1,\ldots,v_{i-1},v_{i+1},\ldots,v_n ]$$
\end{quote}

We now consider the $\ker(\partial)$. This ends up being the cycle space of a graph. They cycles are our tool for determining holes. But currently $\ker(\partial)$, by itself gives a few too many cycles needed to describe capture a hole. So we will define an equivalence class which does:

\begin{quote}
    \textbf{Example}
\end{quote}

\begin{quote}
\end{quote}

\subsection{Homology Groups}

The $n$th homology group is, roughly speaking, a way of finding the number of $n$ dimensional holes in some surface. It is also worth mentioning the $0$th homology group corresponds to the number of connected componenets of a surface. Nevertheless, to say that it "counts" the number of holes is a bit misleading. A circle, for instance, has one $1$-dimensional hole in the center and has one connected component. It is not the case that $H_0(S^1) = 1$ and $H_1(S^1) = 1$. Often one will see the following result instead:

$$H_k(S^1) =
\begin{cases}
    \mathbb{Z}, & k = 0, 1 \\
    \{ 0 \}, & \text{otherwise}
\end{cases}
$$

Sometimes $\mathbb{Z}$ is replaced with $\mathbb{Z}_2$, and in our case we will be mostly concerning ourselves $\mathbb{F}_p$. Although it might not be clear where $\mathbb{Z}$ comes from, one should for now think of $\mathbb{Z}$ as the number $1$, as this coincides with the informal defintion, namely, there is one connected componenet, and one $1$-dimensional hole.

\begin{quote}
    We define the $n$\textbf{th homology group} as:

    $$H_n(\Delta, F_p) = \frac{\ker(\partial_n)}{\text{im}(\partial_{n+1})}$$
\end{quote}

As the homology group is a definition there is no way to "prove" that the $n$th homology group gives us the $n$-dimensional holes. Nevertheless, let us consider an example to gain some for intuition for why this might be a good definition.

\begin{quote}
    \textbf{Example}
\end{quote}

\subsection{Reduced Homology}

We consider a variant of the homology just introduced known as the "Reduced Homology". The reduced homology makes working with certain spaces a bit easier. In particular, for a point the every homology group is $0$ except for $H_0$. We want to make this nicer, so we modify the chain so that every homology group is $0$.

To do this consider we have an arbitrary chain:

$$\cdots \lra C_2 \lra C_1 \lra C_0 \lra 0$$

We then define the function $r$ to be: (straight from wikipedia)
$$r\left(\sum_i n_i \sigma_i\right) = \sum_i n_i$$

And rework the chain to be:

$$\cdots \rightarrow C_2 \rightarrow C_1 \rightarrow C_0 \xrightarrow{r} F_p \rightarrow 0$$

\subsubsection{Acyclic Simplicical Complexes}

Intuitively speaking, a simplicial complex that is \textbf{acyclic} has no holes. More formally:

\begin{quote}
    A simplicial complex is \textbf{acyclic} if the reduced homology is trivial.
\end{quote}

\subsection{Exact Sequence}

We now add an algebra tool to our disposal which will help us formally demonstrate that collapsible simplicial complex are acyclic. We introduce some notation to "visualize" the relationship between the boundary operator and homology groups. We consider any sequence of groups $Y_n$ and a function that maps between them. We write this as:

$$\ldots \longrightarrow Y_2 \rightarrow Y_1 \rightarrow Y_0 \rightarrow 0 $$

So in the case of homology groups:

$$\ldots \rightarrow H_2 \rightarrow H_1 \rightarrow H_0 \rightarrow 0 $$

We can also arrange the lines vertically:

\begin{displaymath}
    \xymatrix{\vdots \ar[d] \\ Y_2 \ar[d] \\ Y_1 \ar[d] \\ Y_0 \ar[d] \\ 0 }
\end{displaymath}

We denote these \textbf{chain complexes} as $Y_{\tb}$. Now we can introduce a useful structure know as a \textbf{short exact sequence}.

\subsection{Useful tools for Homological Algebra}

\begin{quote}
    A \textbf{short exact sequence} is a structure of the form:

    $$0 \lra X_{\tb} \lra Y_{\tb} \lra Z_{\tb} \lra 0 $$

\end{quote}

Why is this useful? For our purposes, short exact sequences are useful because of the following lemma:

\begin{quote}
    \textbf{Lemma A} For a short exact sequence:

    $$0 \lra X_{\tb} \lra Y_{\tb} \lra Z_{\tb} \lra 0 $$

    if $X_{\tb}$ and $Z_{\tb}$ are acyclic, then $Y_{\tb}$ is acyclic.
\end{quote}

This is a well known fact in homological algebra, that we will not prove this fact here. We will, however, use this fact to provide a reasonably rigorous proof that collapsible simplicial complexes are acyclic.


Another useful tool we will not prove:

\begin{quote}
    \textbf{Lemma B} For a short exact sequence:

    $$0 \lra X_{\tb} \lra Y_{\tb} \lra Z_{\tb} \lra 0 $$ 

\end{quote}

\subsubsection{Collapsible Simplicial Complexes are Acyclic}

\section{Fixed Points}

At this point we have two major tools at our disposal:

\begin{enumerate}
    \item{
            All nonevasive properties are collapsible.
        }
    \item{
            All collapsible simplexes are $F_p$ acyclic.
        }
\end{enumerate}


\subsection{Brouwer Fixed Point theorem}

\begin{quote}
    \textbf{Theorem (Brouwer)} Any continuous function from some geometric realization of a simplicial complex $\Delta$ (which we will denote $\ol{\Delta}$) fixes some point. In otherwords, say $f : \ol{\Delta} \to \ol{\Delta}$, then there is some $\delta \in \ol{\Delta}$ such that:
    $$f(\delta) = \delta$$
\end{quote}

We will instead prove a special case of this theorem. Nevertheless, the same general idea can be used to prove this variant of Brouwer's theorem, however, it gets quite a bit messier.

\begin{quote}
    \textbf{Theorem (Brouwer Open Ball)} Any smooth function (continuous and differentiable) function $f$ from the closed (unit) disc to itself has a fixed point.
\end{quote}

\textbf{Proof} We make the following observation: 

\begin{quote}
    \textbf{Retraction Lemma} There is no smooth map $g : D \to \partial D$ such that $g \upharpoonright \partial D \to \partial D$ is equivalent to the identity on the boundary.
\end{quote}

It is often stated as: there is no way to retract the disc to it's boundary.

From this we will construct a function which is smooth but also the identity at the boundary given any smooth $f$ without fixed points.

As $f(x) \neq x$, we can construct a line from $f(x)$ through $x$ to the boundary. By calling this point $g(x)$ we get a new function $g : D \to \partial D$. Now, one important thing to to note is \textit{if} $f$ did indeed have a fixed point, it would be impossible to construct the ray! Now suppose that $x \in \partial D$, then $g(x) = x$ and so $g \upharpoonright \partial D$ is indeed the identity.

All we need to do now is show that $g$ is indeed a smooth function. As they $g(x)$, $f(x)$ and $x$ are all on the same line segment, it follows that:

$$g(x) = tx - (1 - t)f(x)$$

for some $t$. All we need to show is that $t(x)$ is a smooth function. This can be done by observing that $|g(x)| = 1$ so:

$$t^2|x - f(x)|^2 + 2tf(x)[x - f(x)] + f(x)^2 - 1 = 0$$

The solution to the equation can be obtained from the quadratic formula (or WolframAlpha), but the actual solution does not matter, really all that is necessary is to observe that the quadratic formula applied to smooth functions is again smooth and so it follows that $g$ is smoothe \textbf{Retraction Lemma} then implies that $g$ cannot exist. Contradiction.

\section{Abstract Simplicial Complex versus the Geometric Realization}

We make the following observation that there is a geometric realization of $f$. Call it $\ol{f}$. Observe that $\ol{f}$ is a continuous map from the geometric realization of $\Delta$ (we will refer to this as $\ol{\Delta}$ to the geometric realization of $\Delta'$ ($\ol{\Delta}$).

If $\ol{f}$ maps from $\ol{\Delta}$ to $\ol{\Delta}$, it follows that we can apply Brouwer Fixed Point theorem and notice that there is some fixed point $x \in \ol{\Delta}$ such that:

$$\ol{f}(x) = x$$

However, what $f$ does on the simplex is much more limiting than any arbitrary conitinuous function so we can get an even greater set of fixed points!

\begin{quote}
    If $x \in \ol{\Delta}$ then $x$ is the sum of some linear combination of basis vectors. Define the \textbf{support simplex of $x$} $\Delta_x$ to be the collection of non-zero support vectors.
\end{quote}

Because of the way $f$ acts on a simplicial complex, for any the support of any fixed point:

$$f(\Delta_{x}) = \Delta_x$$

This intuitively says:

\begin{quote}
    If $x$ is a fixed point, then the face containing $x$ is fixed by the permutation.
\end{quote}

Furthermore, suppose $\ol{f}$ has some face. Then it may be the case the face was "rotated" from the original $\ol{\Delta}$, however it should be intuitive that this rotation does not affect the center point and so the center point is a fixed point for $f$.

\subsection{Orbits}

The rotations of these faces correspond to the "orbits" of $f$. Observing this fact we can almost conclude the following:

\begin{quote}
    If $P$ is a non-evasive monotone property that invariant under cycles it must be evasive.
\end{quote}

\textbf{Proof}: Suppose otherwise, then $P$ is non-evasive $\implies P$ is collapsible $\implies$ for any function $f$, $f$ has a fixed point. But the orbit of such a function must be $\{1, 2, 3, \ldots |V_0| \}$, the largest face of the graph which means that the completely connected graph does not satisfy the property.

\subsection{Finding the missing cycle}

We are almost ready to prove the main result. In the previous theorem we had the condition that $P$ must be invariant under cycles.

\subsection{Proof of Theorem}

\end{document}
