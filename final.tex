\documentclass[letterpaper,12pt]{article}

\usepackage[table]{xcolor} % Must come before packages that load xcolor

\usepackage{amsfonts}
\usepackage{amsmath}
\usepackage{amssymb}
\usepackage{array}
\usepackage{booktabs}
\usepackage{color}
\usepackage{enumerate}
\usepackage{graphics}
\usepackage{mdwlist}
\usepackage{multicol}
\usepackage[all]{xy}

\usepackage[normalem]{ulem}
\usepackage[margin=1in]{geometry}
\setlength{\parskip}{1ex}

\newcommand{\dom}{\text{dom}}
\newcommand{\ran}{\text{ran}}
\newcommand{\seg}{\text{seg}}
\newcommand{\la}{\langle}
\newcommand{\ra}{\rangle}
\newcommand{\tb}{\text{\textbullet}}

\begin{document}

\title{Fixed Point Theorems and Evasiveness}
\author{Christopher Sumnicht}
\maketitle

\section{Introduction}

In the previous paper, we have concluded that for any simplex associated with a graph property to be nonevasive it must be collapsible. It may still be the case that there is an evasive property which is also collapsible. What then does this property buy us?

We can ask the question: Are there any graph properties which are not collapsible? This is still very much an active field of research, we will focus on one theorem in particular in this paper:

\begin{quote}
    Any nontrivial monotone graph property on a graph with a prime power of vertices is evasive.
\end{quote}

What is the general idea behind such a proof? Show that the associated simplicial complex is not collapsible!

\section{Simplicial Complexes and Algebra}

Before we begin to tackle decision trees we will have to make a detour and develop some tools for working with simplicial complexes. Here is a summary of what we will cover:

\begin{enumerate}
    \item{
            Definition of a Homology Group
        }
    \item{
            Definition of $F_p$ acyclic
        }
    \item{
            $\text{Collapsible} \implies F_p  \text{acyclic}$
        }
\end{enumerate}

\section{Collapsibility and Holes}

We first should provide some intuition as to why collapsibility can (although not nearly as well as homology groups) detect holes in our structure. Imagine that the simplicial complex is a compressible fluid and that the space that is not an element of the complex is solid and rigid. Imagine taking your hand and squishing the fluid in order to reduce it to a single point. If there is a hole, it will feel like a rock in your hand. Otherwise, you can continue compressing everything down to a single point.

Of course, this is simply an analogy. Let's actually prove it.

\subsection{Homology Groups}

The $n$th homology group is, roughly speaking, a way of finding the number of $n$ dimensional holes in some surface. It is also worth mentioning the $0$th homology group corresponds to the number of connected componenets of a surface. Nevertheless, to say that it "counts" the number of holes is a bit misleading. A circle, for instance, has one $1$-dimensional hole in the center and has one connected component. It is not the case that $H_0(S^1) = 1$ and $H_1(S^1) = 1$. Often one will see the following result instead:

$$H_k(S^1) =
\begin{cases}
    \mathbb{Z}, & k = 0, 1 \\
    \{ 0 \}, & \text{otherwise}
\end{cases}
$$

Sometimes $\mathbb{Z}$ is replaced with $\mathbb{Z}_2$, and in our case we will be mostly concerning ourselves $\mathbb{F}_p$. Although it might not be clear where $\mathbb{Z}$ comes from, one should for now think of $\mathbb{Z}$ as the number $1$, as this coincides with the informal defintion, namely, there is one connected componenet, and one $1$-dimensional hole.

\subsubsection{Boundary Maps}

Suppose we want to get the boundary of a simplicial complex. For example if we have a filled triangle $\Delta$ we may want some operator $\partial$ that $\partial \Delta$ gives us just the edges. In the case, a filled triangle should become an empty triangle.

\begin{quote}
    We define the \textbf{Boundary Map} of a simplicial complex to be precisely:
    
    $$d_n([v_0,v_1,\ldots,v_n]) = \sum_{i=0}^n (-1)^i[v_0,v_1,\ldots,v_{i-1},v_{i+1},\ldots,v_n ]$$
\end{quote}

We now consider the $\ker(\partial)$. This ends up being the cycle space of a graph. They cycles are our tool for determining holes. But currently $\ker(\partial)$, by itself gives a few too many cycles needed to describe capture a hole. So we will define an equivalence class which does:

\begin{quote}
    We define the $n$\textbf{th homology group} as:

    $$H_n(\Delta, F_p) = \frac{\ker(\partial_n)}{\text{im}(\partial_{n+1})}$$
\end{quote}

As the homology group is a definition there is no way to "prove" that the $n$th homology group gives us the $n$-dimensional holes. Nevertheless, let us consider an example to gain some for intuition for why this might be a good definition.

\begin{quote}
    \textbf{Example}
\end{quote}

\begin{quote}
\end{quote}

\subsubsection{Acyclic Simplicical Complexes}

Intuitively speaking, a simplicial complex that is \textbf{acyclic} has no holes. More formally states:

\begin{quote}
    A simplicial complex is \textbf{acyclic} if every homology except for the $0$th homology group is trivial.
\end{quote}

\subsubsection{Exact Sequence}

We now add an algebra tool to our disposal which will help us formally demonstrate that collapsible simplicial complex are acyclic. We introduce some notation to "visualize" the relationship between the boundary operator and homology groups. We consider any sequence of groups $Y_n$ and a function that maps between them. We write this as:

\begin{displaymath}
    \xymatrix{\ldots \ar[r] & Y_2 \ar[r] & Y_1 \ar[r] & Y_0 \ar[r] & 0 }
\end{displaymath}

So in the case of homology groups:

\begin{displaymath}
    \xymatrix{\ldots \ar[r] & H_2 \ar[r] & H_1 \ar[r] & H_0 \ar[r] & 0 }
\end{displaymath}

We can also arrange the lines vertically:

\begin{displaymath}
    \xymatrix{\vdots \ar[d] \\ Y_2 \ar[d] \\ Y_1 \ar[d] \\ Y_0 \ar[d] \\ 0 }
\end{displaymath}

We denote these \textbf{chain complexes} as $Y_{\tb}$. Now we can introduce what is referred to as a short exact sequence:

\begin{quote}
    A \textbf{short exact sequence} is a structure of the form:

    \begin{displaymath}
        \xymatrix{0 \ar[r] & X_{\tb} \ar[r] & Y_{\tb} \ar[r] & Z_{\tb} \ar[r] & 0 }
    \end{displaymath}

\end{quote}

Now, this is somewhat of an abuse of notation. as $0 \rightarrow X_{\tb} \rightarrow Y_{\tb} \rightarrow Z_{\tb} \rightarrow 0$ actually becomes: (TODO)

\begin{displaymath}
    \xymatrix{0 \ar[r] & X_{\tb} \ar[r] & Y_{\tb} \ar[r] & Z_{\tb} \ar[r] & 0 }
\end{displaymath}


\subsubsection{Collapsible Simplicial Complexes are Acyclic}

\section{Fixed Points}

At this point we have two major tools at our disposal:

\begin{enumerate}
    \item{
            All nonevasive properties are collapsible.
        }
    \item{
            All collapsible simplexes are $F_p$ acyclic.
        }
\end{enumerate}


\subsection{Lefschetz Fixed Point Theorem}

Now that we have introduced quite a bit of machine to prove a relatively intuitive fact, we can now use this machinery to prove a less obvious fact.

\begin{quote}
    \textbf{Theorem (Lefschetz)}Every simplicial automorphism fixes some subcomplex.
\end{quote}

To put this another way suppose we have some function $f : \Delta \to \Delta$ such that $f$ takes any subcomplex of $\Delta$ to another subcomplex of $\Delta$ there is at least one such subcomplex $\delta$ such that:

$$f(\delta) = \delta$$

Hence $\delta$ is a "fixed point".

\subsubsection{Proof}


\subsection{Proof of Theorem}

\end{document}
